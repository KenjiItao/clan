\documentclass[a4j,12pt]{letter}
\usepackage{amsmath,amssymb}
\usepackage{cite}
\usepackage[dvipdfmx]{graphicx} 
\usepackage{float}
\usepackage{comment}
\usepackage[font=scriptsize]{caption}

\title{Evolution of Kinship Structures Driven by Marriage Tie and Competition}
\author{Kenji Itao, Kunihiko Kaneko}


\begin{document}

Dear Editor:

PNAS

We hereby submit the manuscript entitled "Evolution of Kinship Structures Driven by Marriage Tie and Competition" to be considered for publications in PNAS as a research article.
This paper should be of interest to scientists across a wide spectrum of anthropology and physics.
Kinship structure is a basis of social relationships in indigenous societies and thus has been one of the most important topics in cultural anthropology. Besides, it has gathered much attention of biologists and mathematicians interested in its evolutionary significance and elaboration. Indeed, kinship structures which observed in the field studies satisfy transformation symmetry under marriage and descent relationships, it remains unanswered how they have formed. In this manuscript, we uncover the spontaneous emergence of kinship structures and the transition therein which explains the geographical distribution of kinship structures in the real world.

Families constitute a cultural group, called clan, within which marriage is prohibited, as termed as the incest taboo. The clan attribution governs the mating preference and descent relationships by certain rules, as anthropologists revealed. Such rules form various kinship structures, two of which are focused on---generalized exchange, an indirect exchange of brides among more than two clans, and restricted exchange, a direct exchange of them with the flow of children to different clans. These structures are distributed in different areas and show different cultural consequences. The incest taboo and kinship structures are of social, rather than biological, origin. However, the evolution and transition of the structures are still unknown. Here, we build a model of communities consisting of lineages, family groups, introducing social cooperation among kin and mates and conflict over mating. Each lineage has parameters characterizing the trait and mate preference, which determines the possibility of marriage and the degree of cooperation and conflicts among lineages. By using multi-level selection with population dynamics of lineages and evolution of parameter values, we numerically demonstrate that lineages are clustered in the space of traits and preferences, leading to the emergence of clans with the incest taboo. Generalized exchange emerges when cooperation is highly needed while restricted exchange emerges when the mating conflict is strict.
This is the first model on the emergence of the incest taboo and kinship structures considering social significance of marriage which is consistent with ethnographic records and theoretical studies in anthropology.
Considering the novelty of the model, which is based on nonlinear dynamics, and its applicability to a broad class of anthropology and social evolution, the manuscript should appeal to the wide readership of PNAS, encompassing researchers involved in the fields of physical anthropology and cultural evolution as well as nonlinear and statistical physics.

The suggestion of editor and reviewer (coming soon)

Best regards,

Kenji Itao and Kunihiko Kaneko

\end{document}

