\documentclass[a4j,12pt]{letter}
\usepackage{amsmath,amssymb}
\usepackage{cite}
\usepackage[dvipdfmx]{graphicx} 
\usepackage{float}
\usepackage{comment}
\usepackage[font=scriptsize]{caption}

\title{Evolution of Kinship Structures Driven by Marriage Tie and Competition}
\author{Kenji Itao, Kunihiko Kaneko}


\begin{document}

Dear Editor:

PNAS

We hereby submit the manuscript entitled "Evolution of Kinship Structures Driven by Marriage Tie and Competition" to be considered for publications in PNAS as a research article.
This paper should be of interest to scientists across a wide spectrum of anthropology, evolution, mathematical biology and physics.

Kinship structure is a basis of human social relationships. Elucidation of its structure, in particular in indigeneous society, indeed opened up the field of cultural anthropology, as has been pioneered by the celebrated classic studies by Levi-Strauss.  Through the extensive field-work studies since then, the study of kinship structure remains a core of the anthropology as a basis to understand human families and societies. Besides the anthropologists, the issue of kinship structure gathered some attention of mathematicians, group-theory was introduced to understand which structure is possible.  Little is known theoretically, however, on how kinship structures with incest taboo emerge. Since the publication of Levi-Strauss’ *** of 70 years ago, however,  several theoretical tools and concepts have been developed in the study of complex systems, including simulations of dynamics of multi-agent models, statistical physics for them, and theoretical biology for multi-level evolution.  Time is ripe to answer the classic question on the origin of kinship structure.  In this manuscript, through extensive simulations and theoretical analysis of simple agent-based models with cultural evolution, we have uncovered spontaneous emergence of kinship structures with incest taboo. Different structures as cultural anthropologists classified emergence depending the parameter characterizing the strength of cooperation and competition, which can explain the geographical distribution of kinship structures in the real world.

Families constitute a cultural group, called clan, within which marriage is prohibited, as termed as the incest taboo. The mating preference and descent relationships of clans follow certain rules, as anthropologists revealed. Such rules form two basic structures in kinship ---generalized exchange, an indirect exchange of brides among more than two clans, and restricted exchange, a direct exchange of them with the flow of children to different clans. These structures are distributed in different forms of societies such as agricultural or hunting-gathering one. The incest taboo and kinship structures are believed to be of social, rather than biological, origin. How the different kinship structure emerges and evolves, however, are still unknown. 
Here, we build a model of communities consisting of lineages, i.e., family groups, by introducing social cooperation among kin and mates and conflict over mating, as well as cultural evolution of their traits and mate preference. By using multi-level evolution with population dynamics of lineages and evolution of parameter values, we numerically demonstrate that lineages are clustered in the space of traits and preferences, leading to the emergence of clans with the incest taboo. The generalized exchange emerges when cooperation is highly needed while the restricted exchange emerges when the mating conflict is strict.

This is the first mathematical model that can explain the emergence of the incest taboo and kinship structures, as is also consistent with ethnographic records and theoretical studies in anthropology.
Considering the novelty and universality of our study, with its general applicability to a broad class of anthropology and social evolution, the manuscript should appeal to the wide readership of PNAS, encompassing researchers involved in the fields of anthropology, evolution theory, socio-physics, as well as nonlinear and statistical physics.

The suggestion of editor and reviewer (coming soon)

Best regards,

Kenji Itao and Kunihiko Kaneko

\end{document}

